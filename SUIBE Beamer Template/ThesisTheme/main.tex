%%%%%%%%%%%%%%%%%%%%%%%%%%%%%%%%%%%%%%%%%%%%%%%%%%%%%%%%%%%%%%%%%%%%%%%%%%%%%%%%%%%%%%%%%
% 
% Template for DEE@ISEP Presentations, for LEEC, LETI and MEEC
% developed by Vitor M. R. Cunha - v1.0, Mar 2024
% Suggestions and comments are welcomed (vrc at isep dot ipp dot pt).
%
% Template provided as is, NO SUPPORT will be given.
% NO questions related to Beamer (or LaTeX) will be answered, go to
% https://tug.ctan.org/macros/latex/contrib/beamer/doc/beameruserguide.pdf, or search the web.
%
%%%%%%%%%%%%%%%%%%%%%%%%%%%%%%%%%% CLASS SETTINGS %%%%%%%%%%%%%%%%%%%%%%%%%%%%%%%%%%%%%%

\documentclass[
aspectratio=169,	% Comment for 4:3 aspect ratio
LETI,				% Use this option to select your DEE degree, options: LEEC, LETI
english,			% Select document language, options: portuguese, english
%handout			% Uncomment to make the PDF WITHOUT overlays, e.g., for PDF submission before the presentation
]{DEEclassP}

% Use the 'preamble.tex' file (root folder) do add packages and macros. Keep your main.tex file clean.

%%%%%%%%%%%%%%%%%%%%%%%%%%%%%%%%%%%%%%%%%%%%%%%%%%%%%%%%%% packages
\usepackage{amsmath}		% the main package in the AMS-LATEX distribution
\usepackage{amsfonts}		% extended set of fonts for use in mathematics
\usepackage{amssymb}		% adds new symbols to be used in math mode
\usepackage{mathrsfs}		% math fonts, e.g., Laplace	
\usepackage{float}			% provides the H float modifier option
\usepackage{multirow}		% tables \multirow command
\usepackage{subcaption}		% enables subfigures					 
\usepackage{siunitx}
\usepackage{tikz}
\usepackage{textcomp}
\usepackage{ctex}
\usepackage{gensymb}		% typesetting of units				
\usepackage{verbatim}		% new verbatim environment, \begin{comment}...\end{comment}, \verbatiminput								\						
%add extra packages if needed here
\usepackage{tabularray}
\UseTblrLibrary{booktabs, siunitx}
\usepackage{booktabs,array}
\usepackage{dcolumn}
\usepackage[style=biblatex-juradiss]{biblatex}

\addbibresource{sampleRefs.bib}
\renewcommand*{\bibfont}{\tiny}
%%%%%%%%%%%%%%%%%%%%%%%%%%%%%%%%%%%%%%%%%%%%%%%%%%%%%%%%%% settings
\newcommand\mc[1]{\multicolumn{1}{@{}c@{}}{#1}} % handy shortcut macro
\newcolumntype{R}{>{$}r<{$}} % automatic math mode version of "r" column type

%%%%%%%%%%%%%%%%%%%%%%%%%%%%%%%%%%%%%%%%%%%%%%%%%%%%%%%%%% user macros












%%%%%%%%%%%%%%%%%%%%%%%%%%%%% PRESENTATION INFORMATION %%%%%%%%%%%%%%%%%%%%%%%%%%%%%%%%%%

\title[法學作为科學之無價值性]{\textit{法學作为科學之無價值性}}
\subtitle{\textit{Die Werthlosigkeit der Jurisprudenz als Wissenschaft}\footfullcite{kirchmann1848werthlosigkeit}}
\author[Deinen Namen]{Deinen Namen}	% Your name
\studentnumber{1234567}									% Your student number
%\predate{1 de junho de 2024} 					% Presentation date, comment this line for automatic date (today)

%%%%%%%%%%%%%%%%%%%%%%%%%%%%%%%%%%%%%%%%%%%%%%%%%%%%%%%%%%%%%%%%%%%%%%%%%%%%%%%%%%%%%%%%%
\begin{document}

%\begin{frame}[noframenumbering,plain]
	\maketitlepage
%\end{frame}

\begin{frame}{Sumário}
	\tableofcontents
\end{frame}
%%%%%%%%%%%%%%%%%%%%%%%%%%%%%% MAKE YOUR SLIDES AFTER THIS %%%%%%%%%%%%%%%%%%%%%%%%%%%%%%
%\makesection
\section{Notas Preliminares}

\begin{frame}{Notas Preliminares (I)}

\setlstep{0.3cm}{0cm}{0cm}	% To easily ADD vertical spacing (to standard spacing) between list items.
							% \setlstep{1st depth level}{2nd}{3rd}
\begin{itemize}
	\item Considere que só dispõe de \textcolor{red}{10~minutos} para realizar a apresentação.
	\item Como regra fundamental deverá considerar 1~min por cada slide.
	\begin{itemize}
		\item Assim, não deve preparar mais do que 10 slides.
	\end{itemize}
	\item Não deve usar muito texto nos slides.
	\begin{itemize}
		\item Procure adaptar o texto do relatório às necessidades, necessariamente distintas, da sua apresentação.
	\end{itemize}	
	\item Deve usar imagens, gráficos e esquemas.
	\begin{itemize}
		\item Reutilize os que produziu para o relatório.
	\end{itemize}	
	\item Deve praticar a sua apresentação de forma a garantir que irá transmitir o que pretende.
\end{itemize}

\end{frame}

%%%%%%%%%%%%%%%%%%%%%%%%%%%%%%%%%%%%%%%%%%%%%%%%%%%%%%%%%%%%%%%%%%%%%%%%%%%%%%%%%%%%%%%%%
\begin{frame}[fragile]{Notas Preliminares (II)}
% Use the 'fragile' option when your slide has \verb or a listing

\setlstep{0.3cm}{0cm}{0cm}
\begin{itemize}
	\item A secção \textbf{Sumário} é obrigatória.
	\begin{itemize}
		\item Use os comandos \verb|\section{}| e \verb|\subsection{}| para automaticamente adicionar conteúdo ao Sumário.
		\item No inicio da apresentação deverá oralmente descrever as secções mencionadas no Sumário.
	\end{itemize}	
	\item Também é obrigatória a secção \textbf{Conclusões}.
	\item Se o desejar pode acompanhar a sua apresentação com uma pequena demonstração. 
	\begin{itemize}
		\item Os 10 min de que dispõe podem ser usados ao seu critério, pondere sobre a forma e meios de transmitir a sua mensagem.
	\end{itemize}
	\item Deve procurar ser claro e conciso na sua exposição oral e escrita.
\end{itemize}
\end{frame}

%%%%%%%%%%%%%%%%%%%%%%%%%%%%%%%%%%%%%%%%%%%%%%%%%%%%%%%%%%%%%%%%%%%%%%%%%%%%%%%%%%%%%%%%%
\section{Como Usar o \textit{Template}}
%\subsection{Subsection name}

\begin{frame}[fragile]{Como usar o \textit{template}}

\setlstep{0.3cm}{0cm}{0cm}
\begin{enumerate}
	\item Configuração inicial, no ficheiro \verb|main.tex|: 
	\begin{itemize}
		\item secção \verb|CLASS SETTINGS|: selecionar o formato dos slides, o curso DEE e o idioma.
		\item secção \verb|PRESENTATION INFORMATION|: introduzir o título da apresentação, nome e número do candidato e a data da apresentação.
	\end{itemize}	
	\item Crie os slides (\textit{frames}) após a secção \verb|MAKE YOUR SLIDES AFTER THIS|.
	\item Coloque os ficheiros de imagens no diretório \verb|figures|.
	\item Analise o código deste ficheiro (\verb|main.tex|) onde vai encontrar exemplos e dicas.
\end{enumerate}

\end{frame}

%%%%%%%%%%%%%%%%%%%%%%%%%%%%%%%%%%%%%%%%%%%%%%%%%%%%%%%%%%%%%%%%%%%%%%%%%%%%%%%%%%%%%%%%%
\section{A Classe Beamer}

\begin{frame}{A classe Beamer}

\setlstep{0.5cm}{0cm}{0cm}
\begin{itemize}
	\item Uma apresentação Beamer é criada como qualquer outro documento \LaTeX{}. Beamer é uma classe de \LaTeX{} para criar apresentações que são realizadas usando um projetor.
	\begin{itemize}
		\item O resultado é um ficheiro PDF.
		\item Para a apresentação usa-se um qualquer leitor de PDFs em modo de `página única' e `ecrã completo'.
	\end{itemize}	
	\item Apontadores Beamer:
	\begin{itemize}
		\item \url{https://tug.ctan.org/macros/latex/contrib/beamer/doc/beameruserguide.pdf}
		\item \url{https://www.overleaf.com/learn/latex/Beamer}
	\end{itemize}
\end{itemize}
\end{frame}

%%%%%%%%%%%%%%%%%%%%%%%%%%%%%%%%%%%%%%%%%%%%%%%%%%%%%%%%%%%%%%%%%%%%%%%%%%%%%%%%%%%%%%%%%
\section{Slides Exemplo}

\begin{frame}{Listas}
\framesubtitle{Subtítulo se necessário}

\begin{columns}
\column{0.33\textwidth}

Lista simples:
\begin{itemize}
	\item Item 11
    \begin{itemize}
    	\item Item 21
  	    \item Item 22
  	\end{itemize}
  	\item Item 12
\end{itemize}  	    

\column{0.33\textwidth}
Lista numerada:
\begin{enumerate}
	\item Item 1
    \item Item 2
  	\item Item 3
\end{enumerate} 

\column{0.33\textwidth}
Lista descritiva:
\begin{description}
	\item [Opção 1:] Item 1
	\item [Opção 2:] Item 2
	\item [Opção 3:] Item 3
\end{description}
	
\end{columns}

\end{frame}

%%%%%%%%%%%%%%%%%%%%%%%%%%%%%%
\begin{frame}{Tabelas (I)}

  
\begin{table}[h]							 
	\centering\label{tab1}
        \caption{Exemplo de uma tabela simples 	} 
	\begin{tabular}{ccccc}
	\toprule
	\textbf{coluna 1} & \textbf{coluna 2} & \textbf{coluna 3}\\
	\midrule
		célula 11 & célula 12 & célula 13\\ 
    	célula 21 & célula 22 & célula 23\\  
    	célula 31 & célula 32 & célula 33\\  
	\bottomrule
	\end{tabular}
\end{table}
\end{frame}

\begin{frame}{Tablelas (II)}
\vspace{0.2cm}
\setlength\tabcolsep{3pt} % default value: 6pt
\resizebox{\textwidth}{!}{%
\begin{tabular}{@{} l lr *{13}{R} @{}}
\toprule
&\multicolumn{2}{c}{Settings} & \text{perfect}
&\multicolumn{4}{c}{10\% values} &\multicolumn{4}{c}{30\% values} 
&\multicolumn{4}{c@{}}{50\% values} \\
\cmidrule{2-3} \cmidrule(lr){5-8} \cmidrule(lr){9-12} \cmidrule(l){13-16}
Model & $\beta$ & & & \mc{s1} & \mc{s2} & \mc{s3} & \mc{s4} 
                    & \mc{s1} & \mc{s2} & \mc{s3} & \mc{s4} 
                    & \mc{s1} & \mc{s2} & \mc{s3} & \mc{s4}\\
\midrule
Bias\\ % no need for "\multicolumn{13}{l}{Bias}" ...
&    0  & 336  & -0.006 &
   0.005 & 0.005  & 0.005  & 0.011  & -0.006 & -0.006 & -0.006 & 0.000  & 0.008  & 0.003  & 0.004  & 0.011 \\
&    0.5&  336 & -0.004 &
   0.004 & -0.007 & 0.003  & 0.012  & 0.001  & -0.031 & 0.001  & -0.005 & 0.010  & -0.046 & 0.011  & -0.007 \\
&    1   &  84 & 0.012  &
  -0.004 & -0.024 & -0.003 & 0.005  & 0.021  & -0.046 & 0.019  &  0.004 & 0.038  & -0.071 & 0.035  & -0.071 \\
\addlinespace
MSE\\
&    0  &  336 & 0.026 &
   0.028  & 0.027  & 0.028  & 0.03  & 0.040  & 0.037  & 0.041  & 0.035 & 0.052  & 0.044  & 0.052  & 0.053 \\
&    0.5&  336 & 0.023 &
    0.026  & 0.025  & 0.026  & 0.028 & 0.034  & 0.032  & 0.035  & 0.034 & 0.048  & 0.042  & 0.046  & 0.051 \\
&    1  & 84   & 0.103 &
   0.098  & 0.095  & 0.098  & 0.114 & 0.142  & 0.125  & 0.142  & 0.156 & 0.195  & 0.161  & 0.194  & 0.200 \\
\addlinespace
Coverage\\
&    0  &  336 & 95.2  &
   94.9  & 94.9   & 94.7   & 93.4  & 93.8  & 94.4   & 92.6   & 92.7 & 94.6  & 95.8   & 94.3   & 90.2 \\
&    0.5&  336 & 95.5  &
   95.4  & 95.9   & 94.9   & 93.7 & 94.4  & 94.9   & 94.8   & 92.8 & 95.2  & 95.6   & 95.1   & 88.6 \\
&    1  &   84 & 94.5  &
   96.6  & 96.8   & 96.3   & 94.8 & 96.0  & 95.7   & 95.4   & 92.7 & 96.2  & 96.7   & 96.6   & 92.2 \\
\bottomrule
\end{tabular}}
\begin{table}\label{tab2}
\caption{Exemplo de uma tabela larch}
\end{table}
\end{frame}
	
%%%%%%%%%%%%%%%%%%%%%%%%%%%%%%%%%%%%%%%%%%%%%%%%%%%%%%%%%%%%%%%%%%%%%%%%%%%%%%%%%%%%%%%%%
\begin{frame}{Caixas de Texto}
\footnotesize
\begin{block}{Caixa tipo 1}
Não esquecer que também pode usar referências, e.g.,\footfullcite{larenz1996} .
\end{block}

\begin{alertblock}{Caixa tipo 2}
Por exemplo, uma lista:
\begin{enumerate}
    \item Item 1 
    \item Item 2 
\end{enumerate} 
\end{alertblock}

\begin{examples}
Esta caixa tem um nome pré-definido\footcite{suli2001}. 
\end{examples}

\begin{theorem}
    falsa demonstratio Regel\footfullcite{ZFLT202201012}
\end{theorem}
    
\end{frame}

%%%%%%%%%%%%%%%%%%%%%%%%%%%%%%%%%%%%%%%%%%%%%%%%%%%%%%%%%%%%%%%%%%%%%%%%%%%%%%%%%%%%%%%%%
\begin{frame}{Equações}
  	
\begin{itemize}
	\item Equações alinhadas e numeradas:
	\begin{align}
		\frac{dx_1}{dt} &= f_1\left(x_1,x_2\right)\\
		\frac{dx_2}{dt} &= f_2\left(x_1,x_2\right)
	\end{align}
	\vspace{0.3cm}
	\item  Equação (não numerada, em modo \textit{display}): 	
	$$e = m \cdot c^2$$
\end{itemize}  	    

\end{frame}

%%%%%%%%%%%%%%%%%%%%%%%%%%%%%%%%%%%%%%%%%%%%%%%%%%%%%%%%%%%%%%%%%%%%%%%%%%%%%%%%%%%%%%%%%
\begin{frame}[fragile]{Código}

\begin{center}
\begin{minipage}{0.85\textwidth}
\begin{block}{\small{Exemplo de uma listagem}}
\begin{lstlisting}[language=Python]
for number in range(1,11):    # Loop from 1 to 10
    if number % 2 == 0:       # Check if the number is even
        print(f"{number} *")  # Print even numbers
    else:
        print(number)         # Print odd numbers
\end{lstlisting}
\end{block}
\end{minipage}
\end{center}

\end{frame}

%%%%%%%%%%%%%%%%%%%%%%%%%%%%%%
\begin{frame}%[plain]
\frametitle{Figura Grande}

\begin{figure}[h]						
	\centering							
	\includegraphics[width=\textwidth,height=0.8\textheight,keepaspectratio]{figures/sample.png}
\end{figure}

% Link to an extermnal video file in the PC. The default video player will start with the video.
%\begin{itemize}
%	\item Vídeo externo 	
%\end{itemize}
%\movie[externalviewer]{\includegraphics[scale=0.3]{videothumb.png}}{video_sample.mp4}

% Link to an extermnal video in Youtube. Opens a brower in the video link. Can be used to open any external resource.
%\begin{figure}[h]						
%	\centering							
%	\href{https://youtu.be/LZCEzHhl0Jc?si=KSKEW3uwAEos4rMa}{\includegraphics[scale=0.3]{sample.png}}
%\end{figure}

\end{frame}

%%%%%%%%%%%%%%%%%%%%%%%%%%%%%%%%%%%%%%%%%%%%%%%%%%%%%%%%%%%%%%%%%%%%%%%%%%%%%%%%%%%%%%%%%
\begin{frame}[fragile]{\textit{Overlays} (I)}

Comando \verb|\pause| numa lista:
\begin{itemize}
	\pause
	\item Item 1
	\pause
	\item Item 2
	\pause
  	\item Item 3
\end{itemize} 

\end{frame}
%%%%%%%%%%%%%%%%%%%%%%%%%%%%%%%%%%%%%%%%%%%%%%%%%%%%%%%%%%%%%%%%%%%%%%%%%%%%%%%%%%%%%%%%%
\begin{frame}[fragile]{\textit{Overlays} (II)}	

Alternativamente, usando \textit{overlay specifications} onde se especifica o(s) slide(s) em que um elemento é visível:
\begin{itemize}
	\item<4-> Item 1
	\item<3-> \textcolor<5>{red}{Item 2}
  	\item<2> Item 3
\end{itemize}

\end{frame}

%%%%%%%%%%%%%%%%%%%%%%%%%%%%%%%%%%%%%%%%%%%%%%%%%%%%%%%%%%%%%%%%%%%%%%%%%%%%%%%%%%%%%%%%%
\begin{frame}[fragile]{\textit{Overlays} (III)}	

\begin{itemize}
	\item Existem comandos específicos para definir a visibilidade de elementos:
	\begin{itemize}
		\item \verb|\onslide|, \verb|\visible|, \verb|\invisible|, \verb|\only| e \verb|\alt|.
	\end{itemize}
	\item As \textit{overlay specifications} podem ser aplicadas em diversas situações definidas pela classe Beamer. Assim é possível, por exemplo, sobrepor figuras:
\end{itemize}

\begin{figure}[h]						
	\centering							
	\includegraphics<1|handout:0>[scale=0.6]{imgoverlay1.pdf}%
	\includegraphics<2>[scale=0.6]{imgoverlay2.pdf}%
\end{figure}

\end{frame}
%%%%%%%%%%%%%%%%%%%%%%%%%%%%%%%%%%%%%%%%%%%%%%%%%%%%%%%%%%%%%%%%%%%%%%%%%%%%%%%%%%%%%%%%%

%%%%%%%%%%%%%%%%%%%%%%%%%%%%%%%%%%%%%%%%%%%%%%%%%%%%%%%%%%%%%%%%%%%%%%%%%%%%%%%%%%%%%%%%%
\begin{frame}{Colunas}

\setlstep{0.3cm}{0cm}{0cm}
\begin{columns}
\column{0.5\textwidth}
  	\begin{itemize}
  	  	\item Slide exemplo com duas colunas (cada uma com 50\% da largura do slide).
  		\item A soma das larguras das colunas deverá ser sempre 100\% para manter o alinhamento do conteúdo de todos os slides.
  		\begin{itemize}
  			\item Se necessário, usar colunas vazias.
  		\end{itemize}
  		\item Podem ser usados todos os tipos de ambientes dentro de uma coluna: lista, figura, tabela, caixa, $\ldots$
	\end{itemize}
 
\column{0.5\textwidth}
\begin{figure}[h]						
	\centering							
	%\includegraphics[scale=0.5]{sample.png}
	\includegraphics[width=0.8\textwidth]{sample.png}
\end{figure}
\end{columns}

\end{frame}

%%%%%%%%%%%%%%%%%%%%%%%%%%%%%%%%%%%%%%%%%%%%%%%%%%%%%%%%%%%%%%%%%%%%%%%%%%%%%%%%%%%%%%%%%
\section{A Apresentação}
\begin{frame}[fragile]{Considerações para o dia da apresentação}

\setlstep{0.3cm}{0cm}{0cm}
\begin{itemize}
	\item Estar presente com a antecedência necessária para preparar a apresentação.
	\item Abrir o PDF da apresentação com o Adobe Acrobat Reader, Foxit Reader, entre outros.
	\begin{itemize}
		\item Evitar usar leitores de PDFs de \textit{browsers}.
	\end{itemize} 
	\item Colocar em modo de ecrã inteiro, e.g.:
	\begin{itemize}
		\item Adobe Acrobat Reader: \verb|CTRL + L|
		\item Foxit Reader: \verb|F11|
	\end{itemize}	
	\item Deve ter consigo uma cópia do relatório para a fase de discussão.
\end{itemize}
	
\end{frame}

%%%%%%%%%%%%%%%%%%%%%%%%%%%%%%%%%%%%%%%%%%%%%%%%%%%%%%%%%%%%%%%%%%%%%%%%%%%%%%%%%%%%%%%%%
\section{Conclusões}

\begin{frame}{Conclusões}

\setlstep{0.5cm}{0cm}{0cm}
\begin{itemize}
	\item Esta secção é obrigatória e destina-se à síntese das principais conclusões.

	\item Quais são as consequências e a relevância do trabalho realizado.

	\item Perspetive futuros desenvolvimentos. 
\end{itemize}
	
\end{frame}

%%%%%%%%%%%%%%%%%%%%%%%%%%%%%%%%%%%%%%%%%%%%%%%%%%%%%%%%%%%%%%%%%%%%%%%%%%%%%%%%%%%%%%%%%

\begin{frame}{~}
  \transboxin	% Slide transition effect, there are several options. Check Beamer user guide.
				% Different PDF viewers have different interpretations of these commands, always test beforehand
  \centering
  \Large %\textit{通过教义法学,超越教义法学}\\~
  
  %\textit{Durch d.Rechtsdogmatik, über d.Rechtsdogmatik hinaus. }\\~

\textit{\textbf{Danke für ihren Aufmerksamkeit!}}
\end{frame}

%%%%%%%%%%%%%%%%%%%%%%%%%%%%%%%%%%%%%%%%%%%%%%%%%%%%%%%%%%%%%%%%%%%%%%%%%%%%%%%%%%%%%%%%%
% Print the bibliographic references, comment this frame if not needed

%\begin{frame}[t]

    %\printbibliography


%\begin{columns}
%column{0.33\textwidth}
    %\makereferencesframegerman{sampleRefs}% Use the .bib from your report
%\column{0.33\textwidth}
%\end{columns}
%\begin{frame}[t,allowframebreaks]		% Use this line and comment previous if references list takes more than one slide
		
%\end{frame}

%%%%%%%%%%%%%%%%%%%%%%%%%%%%%%%%%%%%%%%%%%%%%%%%%%%%%%%%%%%%%%%%%%%%%%%%%%%%%%%%%%%%%%%%%
\end{document}